% CLASSES
% Classe padrão
\documentclass[12pt, a4paper, openright, titlepage, oneside]{book}

% PACOTES
\usepackage[T1]{fontenc}
\usepackage[utf8]{inputenc}
\usepackage[brazilian]{babel}
\usepackage{epsfig}
\usepackage{eso-pic}
\usepackage{subfigure}
\usepackage{amsfonts}
\usepackage{amsmath}
\usepackage{amssymb}
\usepackage[thmmarks,amsmath]{ntheorem}
\usepackage{boxedminipage}
\usepackage{geometry}
\usepackage{theorem}
\usepackage{fancybox}
\usepackage{fancyhdr}
\usepackage{ifthen}
\usepackage{url}
\usepackage{afterpage}
\usepackage{color}
\usepackage{colortbl}
\usepackage{rotating}
\usepackage{makeidx}
\usepackage{epstopdf}
\usepackage{indentfirst}
\usepackage{times}
\usepackage{helvet}
\usepackage{soul}
\usepackage{float}
\usepackage{listings}
\usepackage{listingsutf8}
\usepackage[pdfstartview=FitH]{hyperref}
\usepackage[table,xcdraw]{xcolor}
\usepackage{multirow}
\usepackage{pdfpages}
\usepackage{lastpage}
\usepackage{lipsum}
\usepackage[linesnumbered,lined,boxruled,commentsnumbered,algochapter,portuguese]{algorithm2e}
\usepackage[framed,numbered]{config/mcode}
\usepackage{array}
\usepackage{tabularx}
\usepackage[font=normalsize]{caption}
\usepackage{lastpage}
\usepackage{config/ft_unb} % segue padrão de fonte Times
\usepackage[num,abnt-etal-list=0]{config/abntex2/abntex2cite} % Citações pela ABNT
\usepackage{nomencl} % Gerar lista de abreviações - http://franz.kollmann.in/latex/latex.html

\newcommand\BackgroundPic{%
    \put(0,0){%
    \parbox[b][\paperheight]{\paperwidth}{%
    \vfill
    \centering
    \includegraphics[width=\paperwidth,height=\paperheight,%
    keepaspectratio]{figuras/capa_fundo.jpg}%
    \vfill
}}}

\newcolumntype{C}[1]{>{\centering\let\newline\\\arraybackslash\hspace{0pt}}m{#1}}
\newcolumntype{L}[1]{>{\raggedright\let\newline\\\arraybackslash\hspace{0pt}}m{#1}}
\newcolumntype{R}[1]{>{\raggedleft\let\newline\\\arraybackslash\hspace{0pt}}m{#1}}

\hypersetup{colorlinks,
citecolor=black,
filecolor=black,
linkcolor=black,
urlcolor=black,
}

\newcommand{\citeC}[1]{[\citeonline{#1}]}

\definecolor{codegreen}{rgb}{0,0.6,0}
\definecolor{codegray}{rgb}{0.5,0.5,0.5}
\definecolor{codepurple}{rgb}{0.58,0,0.82}
\lstdefinestyle{code_style}{
    commentstyle=\color{codegreen},
    keywordstyle=\color{magenta},
    numberstyle=\tiny\color{codegray},
    stringstyle=\color{orange},
    basicstyle=\ttfamily\footnotesize,
    breakatwhitespace=false,         
    breaklines=true,                 
    captionpos=b,                    
    keepspaces=true,                 
    numbers=left,                    
    numbersep=5pt,                  
    showspaces=false,                
    showstringspaces=false,
    showtabs=false,                  
    tabsize=2,
    stepnumber=1,
    firstnumber=1,
    numberfirstline=true,
    alsodigit={.:;},
    extendedchars=true,
    literate=
        {á}{{\'a}}1
        {à}{{\`a}}1
        {ã}{{\~a}}1
        {é}{{\'e}}1
        {ê}{{\^e}}1
        {í}{{\'i}}1
        {ó}{{\'o}}1
        {õ}{{\~o}}1
        {ú}{{\'u}}1
        {ü}{{\"u}}1
        {ç}{{\c{c}}}1
}
\lstset{style=code_style}

\lstdefinelanguage{CSS}{
    keywords={
        color,background-image:,margin,padding,font,weight,display,position,top,left,right,bottom,list,style,border,size,white,space,min,width, transition:, transform:, transition-property, transition-duration, transition-timing-function,color:,font-size:,
    },
    ndkeywords={
        accelerator,azimuth,background,background-attachment,
        background-color,background-image,background-position,
        background-position-x,background-position-y,background-repeat,
        behavior,border,border-bottom,border-bottom-color,
        border-bottom-style,border-bottom-width,border-collapse,
        border-color,border-left,border-left-color,border-left-style,
        border-left-width,border-right,border-right-color,
        border-right-style,border-right-width,border-spacing,
        border-style,border-top,border-top-color,border-top-style,
        border-top-width,border-width,bottom,caption-side,clear,
        clip,color,content,counter-increment,counter-reset,cue,
        cue-after,cue-before,cursor,direction,display,elevation,
        empty-cells,filter,float,font,font-family,font-size,
        font-size-adjust,font-stretch,font-style,font-variant,
        font-weight,height,ime-mode,include-source,
        layer-background-color,layer-background-image,layout-flow,
        layout-grid,layout-grid-char,layout-grid-char-spacing,
        layout-grid-line,layout-grid-mode,layout-grid-type,left,
        letter-spacing,line-break,line-height,list-style,
        list-style-image,list-style-position,list-style-type,margin,
        margin-bottom,margin-left,margin-right,margin-top,
        marker-offset,marks,max-height,max-width,min-height,
        min-width,-moz-binding,-moz-border-radius,
        -moz-border-radius-topleft,-moz-border-radius-topright,
        -moz-border-radius-bottomright,-moz-border-radius-bottomleft,
        -moz-border-top-colors,-moz-border-right-colors,
        -moz-border-bottom-colors,-moz-border-left-colors,-moz-opacity,
        -moz-outline,-moz-outline-color,-moz-outline-style,
        -moz-outline-width,-moz-user-focus,-moz-user-input,
        -moz-user-modify,-moz-user-select,orphans,outline,
        outline-color,outline-style,outline-width,overflow,
        overflow-X,overflow-Y,padding,padding-bottom,padding-left,
        padding-right,padding-top,page,page-break-after,
        page-break-before,page-break-inside,pause,pause-after,
        pause-before,pitch,pitch-range,play-during,position,quotes,
        -replace,richness,right,ruby-align,ruby-overhang,
        ruby-position,-set-link-source,size,speak,speak-header,
        speak-numeral,speak-punctuation,speech-rate,stress,
        scrollbar-arrow-color,scrollbar-base-color,
        scrollbar-dark-shadow-color,scrollbar-face-color,
        scrollbar-highlight-color,scrollbar-shadow-color,
        scrollbar-3d-light-color,scrollbar-track-color,table-layout,
        text-align,text-align-last,text-decoration,text-indent,
        text-justify,text-overflow,text-shadow,text-transform,
        text-autospace,text-kashida-space,text-underline-position,top,
        unicode-bidi,-use-link-source,vertical-align,visibility,
        voice-family,volume,white-space,widows,width,word-break,
        word-spacing,word-wrap,writing-mode,z-index,zoom, font-family:
    },
    sensitive=true,
    morecomment=[l]{//},
    morecomment=[s]{/*}{*/},
    morestring=[b]',
    morestring=[b]",
    alsoletter={:},
    alsodigit={-}
}

\lstdefinelanguage{JavaScript}{
    morekeywords={
        typeof, new, true, false, catch, function, return, null, catch, switch, var, if, in, while, do, else, case, break
    },
    morecomment=[s]{/*}{*/},
    morecomment=[l]//,
    morestring=[b]",
    morestring=[b]'
}

\lstdefinelanguage{HTML5}{
    language=html,
    sensitive=false,	
    alsoletter={<>=-},	
    morecomment=[s]{<!-}{-->},
    tag=[s],
    keywords={
        <, </, >,
        </a, <a, </a>,
        </abbr, <abbr, </abbr>,
        </address, <address, </address>,
        </area, <area, </area>,
        </area, <area, </area>,
        </article, <article, </article>,
        </aside, <aside, </aside>,
        </audio, <audio, </audio>,
        </audio, <audio, </audio>,
        </b, <b, </b>,
        </base, <base, </base>,
        </bdi, <bdi, </bdi>,
        </bdo, <bdo, </bdo>,
        </blockquote, <blockquote, </blockquote>,
        </body, <body, </body>,
        </br, <br, </br>,
        </button, <button, </button>,
        </canvas, <canvas, </canvas>,
        </caption, <caption, </caption>,
        </cite, <cite, </cite>,
        </code, <code, </code>,
        </col, <col, </col>,
        </colgroup, <colgroup, </colgroup>,
        </data, <data, </data>,
        </datalist, <datalist, </datalist>,
        </dd, <dd, </dd>,
        </del, <del, </del>,
        </details, <details, </details>,
        </dfn, <dfn, </dfn>,
        </div, <div, </div>,
        </dl, <dl, </dl>,
        </dt, <dt, </dt>,
        </em, <em, </em>,
        </embed, <embed, </embed>,
        </fieldset, <fieldset, </fieldset>,
        </figcaption, <figcaption, </figcaption>,
        </figure, <figure, </figure>,
        </footer, <footer, </footer>,
        </form, <form, </form>,
        </h1, <h1, </h1>,
        </h2, <h2, </h2>,
        </h3, <h3, </h3>,
        </h4, <h4, </h4>,
        </h5, <h5, </h5>,
        </h6, <h6, </h6>,
        </head, <head, </head>,
        </header, <header, </header>,
        </hr, <hr, </hr>,
        </html, <html, </html>,
        </i, <i, </i>,
        </iframe, <iframe, </iframe>,
        </img, <img, </img>,
        </input, <input, </input>,
        </ins, <ins, </ins>,
        </kbd, <kbd, </kbd>,
        </keygen, <keygen, </keygen>,
        </label, <label, </label>,
        </legend, <legend, </legend>,
        </li, <li, </li>,
        </link, <link, </link>,
        </main, <main, </main>,
        </map, <map, </map>,
        </mark, <mark, </mark>,
        </math, <math, </math>,
        </menu, <menu, </menu>,
        </menuitem, <menuitem, </menuitem>,
        </meta, <meta, </meta>,
        </meter, <meter, </meter>,
        </nav, <nav, </nav>,
        </noscript, <noscript, </noscript>,
        </object, <object, </object>,
        </ol, <ol, </ol>,
        </optgroup, <optgroup, </optgroup>,
        </option, <option, </option>,
        </output, <output, </output>,
        </p, <p, </p>,
        </param, <param, </param>,
        </pre, <pre, </pre>,
        </progress, <progress, </progress>,
        </q, <q, </q>,
        </rp, <rp, </rp>,
        </rt, <rt, </rt>,
        </ruby, <ruby, </ruby>,
        </s, <s, </s>,
        </samp, <samp, </samp>,
        </script, <script, </script>,
        </section, <section, </section>,
        </select, <select, </select>,
        </small, <small, </small>,
        </source, <source, </source>,
        </span, <span, </span>,
        </strong, <strong, </strong>,
        </style, <style, </style>,
        </summary, <summary, </summary>,
        </sup, <sup, </sup>,
        </svg, <svg, </svg>,
        </table, <table, </table>,
        </tbody, <tbody, </tbody>,
        </td, <td, </td>,
        </template, <template, </template>,
        </textarea, <textarea, </textarea>,
        </tfoot, <tfoot, </tfoot>,
        </th, <th, </th>,
        </thead, <thead, </thead>,
        </time, <time, </time>,
        </title, <title, </title>,
        </tr, <tr, </tr>,
        </track, <track, </track>,
        </u, <u, </u>,
        </ul, <ul, </ul>,
        </var, <var, </var>,
        </video, <video, </video>,
        </wbr, <wbr, </wbr>,
        />, <!
    },
    otherkeywords={
        % General
        >,
        % Standard tags
        <!DOCTYPE,
        </html, <html, <head, <title, </title, <style, </style, <link, </head, <meta, />, <h1, <h2, </h1, </h2,
        % body
        </body, <body,
        % Divs
        </div, <div, </div>, 
        % Paragraphs
        </p, <p, </p>,
        % scripts
        </script, <script,
        % More tags...
        <canvas, /canvas>, <svg, <rect, <animateTransform, </rect>, </svg>, <video, <source, <iframe, </iframe>, </video>, <image, </image>, <header, </header, <article, </article
    },
    ndkeywords={
        % General
        =,
        % HTML attributes
        charset=, src=, id=, width=, height=, style=, type=, rel=, href=,
        % SVG attributes
        fill=, attributeName=, begin=, dur=, from=, to=, poster=, controls=, x=, y=, repeatCount=, xlink:href=,
        % properties
        margin:, padding:, background-image:, border:, top:, left:, position:, width:, height:, margin-top:, margin-bottom:, font-size:, line-height:,
        % CSS3 properties
        transform:, -moz-transform:, -webkit-transform:,
        animation:, -webkit-animation:,
        transition:,  transition-duration:, transition-property:, transition-timing-function:,
    }
}
\makeindex
\makenomenclature

\makeatletter
\hypersetup{
    pdftoolbar=true,
    pdfmenubar=true,
    pdffitwindow=false,
    pdfstartview={FitH},
    pdftitle={Análise exploratória de uma conversa do WhatsApp entre duas pessoas},
    pdfauthor={Alunos},
    pdfsubject={Monografia Engenharia Elétrica - UnB},
    pdfcreationdate={27/09/2022}
}
\makeatother

% Define acrônimos
\nomenclature{CSS}{\textit{Cascading Style Sheets} - Folhas de Estilo em Cascatas}
\nomenclature{DNS}{\textit{Domain Name System} - Sistema de nome de domínio}
\nomenclature{HTML}{\textit{HiperText Markup Language} - Linguagem de Marcação de Hipertexto}
%%%%%%%%%%%%%%%%%%%%%%%%%%%%%%%%%%%%%%%%%%%%%%%%%%%%%%%%%%%%%%%

% DOCUMENTO PRINCIPAL
\begin{document}

\setcounter{secnumdepth}{3} % numeração de seções até nível 3
\setcounter{tocdepth}{2} % numeração de seções no sumário até nível 2
\pagestyle{empty}

% GRAU
\grau{Engenheiro}{Eletricista} \tipodemonografia{o}{Trabalho de Conclusão de Curso}

% TÍTULO
% Os comandos a seguir servem para definir o título do trabalho. Para evitar  que o latex defina automaticamente a quebra de linha, foram definidos um comando por linha. Desta forma o autor define como quer que o título seja dividido em várias linhas. O exemplo abaixo é para um título que ocupa três linhas. Observe que mesmo com a linha 4 não sendo utilizada, o comando \titulolinhaiv é chamado.

\titulolinhai{Análise exploratória de uma conversa }
\titulolinhaii{do WhatsApp entre duas pessoas}
\titulolinhaiii{}
\titulolinhaiv{}

% AUTORES
\autori{Aluno 1}
\autorii{Aluno 2}
\autoriii{}

% BANCA EXAMINADORA
% Os nomes dos membros da banca são definidos a seguir. Pode-se ter até 5 membros da banca, numerados de i a v (algarismos romanos).
% Para trabalhos com apenas um autor, deve-se usar \autorii{} para que não apareça um nome para segundo autor. É incumbência do usuário definir no argumento dos comandos a afiliação do membro da banca, assim como sua posição (se for orientador ou co-orientador). 
% Os nomes definidos pelos comandos abaixo aparecem na ordem de i a v.
\membrodabancai{Prof. 1, ENE/UnB}
\membrodabancaifuncao{Orientador}
\membrodabancaii{Prof. 2, ENE/UnB}
\membrodabancaiifuncao{Examinador interno}
\membrodabancaiii{Prof. 3, ENE/UnB}
\membrodabancaiiifuncao{Examinador interno}

% DATA DA DEFESA
\mes{Setembro}
\ano{2022}

% FICHA CATALOGRÁFICA
% Colocar o nome do autor como vai aparecer no catálogo. Último sobrenome primeiro, depois o nome e sobrenomes intermediários. Ex.: Borges, Geovany Araújo
\autorcatalogo{Oliveira, Matheus Noschang de; Braz, Rubens Saito Mira}
% Colocar o nome abreviado. Último sobrenome primeiro, depois as iniciais do nome e sobrenomes intermediários. Ex.: Borges, G.A.
\autorabreviadocatalogo{Oliveira, M.N.; Braz, R.S.M}

% PALAVRAS CHAVE
\palavraschavecatalogoi{Análise exploratória}
\palavraschavecatalogoii{WhatsApp}
\palavraschavecatalogoiii{Visualização de dados}
\palavraschavecatalogoiv{Python}

% NÚMERO DA PUBLICAÇÃO
% Fornecido pelo departamento após a defesa
%\publicacao{TCC-}

%Número de páginas da dissertação.
\numeropaginascatalogo{\pageref{LastPage}~p.}

% Comandos para criar a capa e a página de assinaturas
\capaprincipal
\capaassinaturas
\setcounter{page}{3}

% Comando para criar a ficha catalográfica
\fichacatalografica
\frontmatter
\fontsize{12}{14}\selectfont

% Comando para criar a página de dedicatória
\dedicatoriaautori{Agradecimentos 1}

\dedicatoriaautorii{Agradecimentos 2}
\dedicatoria

% Comando para criar a página de agradecimentos
\agradecimentosautori{\hspace{1cm}A Deus por ter me dado saúde e sanidade para enfrentar os mais diversos desafios e percalços...}

\agradecimentosautorii{\hspace{1cm}Gostaria de agradecer inicialmente ao meu grande amigo Matheus Noschang...}



\agradecimentos

\setcounter{page}{6}

\resumo{Resumo}{\hspace{1cm}A análise de dados vem se mostrando extremamente útil na identificação de padrões e informações pertinentes que permitem a tomada de decisões assertivas.

Palavras Chave: análise exploratória; whatsapp; visualização de dados; python.
}

\vspace*{2cm}

\resumo{Abstract}{\hspace{1cm}Data analysis has been proving itself extremely useful for identifying patterns and pertinent information that allows assertive decision making.

Keywords: exploratory analysis; whatsapp; data visualization; python.}

% Sumário e listas
\sumario
\listadefiguras
\listadetabelas
\listadealgoritmos
\listadeacronimos

\mainmatter
\setcounter{page}{1} \pagenumbering{arabic} \pagestyle{plain}

% Inclua capitulos da dissertação aqui
\chapter{Introdução}

As inúmeras transformações digitais e a pandemia de Covid-19 fizeram com que as dinâmicas de comunicação interpessoal sofressem profundas alterações e adaptações para enfrentarmos os novos desafios. Diferentes mensageiros vêm ganhando cada vez mais usuários e se consolidando no mercado, sendo o atual líder o WhatsApp. De acordo com pesquisa realizada pela Opinion Box em parceria com o site de notícias Mobile Time, 99\% dos brasileiros que possuem \textit{smartphones} têm o referido aplicativo instalado \citeC{costa_pirataria_2001}.

\section{Definição do problema}

Apesar do WhatsApp ser o mensageiro mais utilizado no Brasil, o aplicativo não supre as necessidades de informações estatísticas sobre uma conversa pessoal, emergindo daí a motivação para a elaboração deste trabalho. Além disso, as ferramentas existentes que propõem-se a resolver este problema apresentam visualizações simples e estatísticas tanto quanto rasas como resultado de suas análises, como por exemplo: gráficos mal formatados e com cores destoantes, poucas informações oferecidas, ausência de estatística de tempo gasto na conversa, etc. Portanto, detectou-se espaço para melhoria.
\chapter{Fundamentação teórica}

\resumodocapitulo{Este capítulo apresenta um resumo sobre os fundamentos teóricos utilizados como referência para este trabalho. Também são apresentados trabalhos relacionados e um descritivo da infraestrutura desenvolvida para a criação do site \href{https://analizap.tk}{Analizap}.}

\section{Infraestrutura do projeto}

\subsection{Domínio}

Todo site precisa de uma infraestrutura básica para estar online e um dos aspectos mais importantes e lembrados pelos usuários é o nome do domínio, que nada mais é do que o endereço digitado no navegador de internet para acessar a página desejada. O domínio do site aqui apresentado é \url{https://analizap.tk}.
\chapter{Metodologia}

\resumodocapitulo{Este capítulo versa sobre as metodologias aplicadas para o desenvolvimento do projeto. Primeiramente implementou-se um protótipo utilizando o Jupyter Notebook para facilitar a visualização e desenvolvimento dos códigos \textit{Python}. Em seguida, segmentou-se o código em quatro grandes arquivos - \textit{parsers}, \textit{visualizations}, \textit{dataframe-generator} e \textit{statistics} - que conectam-se diretamente ao \textit{front-end} (desenvolvido na mesma linguagem), onde são apresentados os resultados numa página web.}

\section{Introdução}

Inicialmente realizou-se uma pesquisa de mercado para levantar as ferramentas existentes que fazem análise de conversas do WhatsApp. Para a busca de sites foi utilizado o Google, para aplicativos, a Google Play Store (Android) e App Store (iOS) e os trabalhos acadêmicos foram encontrados utilizando o Google Scholar. Foi feito um apanhado das principais e mais relevantes informações apresentadas pelos artigos e serviços relacionados, compiladas na lista a seguir:

\chapter{Resultado}

\resumodocapitulo{Aqui expõem-se os resultados obtidos pelos algoritmos desenvolvidos neste trabalho a fim de exemplificar a proposta do produto desenvolvido.}

O resultado obtido ao final do projeto foi um produto que objetiva informar o usuário sobre curiosidades e fatos de uma conversa do WhatsApp entre duas pessoas com total segurança, anonimidade e de forma gratuita. O tempo de processamento para gerar as estatísticas em todos os casos testados foi inferior a 1 minuto (conversas com até 200 mil mensagens), sendo todos os gráficos interativos, permitindo a função \textit{zoom}, exportação do gráfico como imagem e, nos casos em que se aplica, a filtragem dos dados exibidos no gráfico.

O site \href{http://analizap.tk/}{Analizap} conta com uma \href{https://analizap.tk/exemplo}{página de exemplo} interativa de resultados que demonstra o produto final entregue aos usuários do serviço. Para obter os resultados para uma conversa pessoal, é preciso exportar o \textit{chat} escolhido e submeter o documento de texto no formulário apresentado na Figura \ref{formulario_txt}, disponível na página inicial do site.

\begin{figure}[H]
    \centering
    \includegraphics[scale=0.5]{TCC/figuras/form.png}
    \caption{Formulário de envio da conversa a ser analisada no site Analizap.}
    \label{formulario_txt}
\end{figure}

Para ilustrar o resultado final com casos reais, foram selecionadas três conversas previamente autorizadas com participantes que configuram classes distintas de conversas: relacionamento amoroso, profissional e familiar. As capturas de tela do site podem ser vistas abaixo para cada uma das classes de conversa. Por fim, é apresentada uma análise sobre as características intrínsecas de cada tipo de conversa e algumas diferenças entre elas.

Um detalhe importante é que o gráfico de palavras mais utilizadas não é apresentado por privacidade. Além disso, no gráfico intitulado “Quantidade de mensagens e tempo de conversa por dia”, nos casos de uma conversa profissional e familiar, os exemplos utilizados são de conversas de muitos meses e, por isso, apresentamos apenas uma estatística para facilitar a visualização das informações: a quantidade de mensagens enviadas (barras laranjadas). Não são exibidas as estatísticas “Minutos” (tempo gasto na conversa, barras azuis) e a “Mensagens por minuto” (velocidade das mensagens trocadas, linha cinza escuro).

\begin{enumerate}
  \item \textbf{Relacionamento amoroso:}
    \begin{figure}[H]
        \centering
        \includegraphics[scale=0.2]{TCC/figuras/ds-lu.jpg}
        \caption{Estatísticas descritivas de namorados.}
        \label{ds-namorados}
    \end{figure}
    
    \begin{figure}[H]
        \centering
        \includegraphics[scale=0.2]{TCC/figuras/pz-lu.jpg}
        \caption{Gráficos de pizza de namorados.}
        \label{pz-namorados}
    \end{figure}
    
    \begin{figure}[H]
        \centering
        \includegraphics[scale=0.18]{TCC/figuras/bars-lu.jpg}
        \caption{Série temporal de namorados.}
        \label{bars-namorados}
    \end{figure}
    
    \begin{figure}[H]
        \centering
        \includegraphics[scale=0.18]{TCC/figuras/area-lu.jpg}
        \caption{Gráfico mensal de namorados.}
        \label{area-namorados}
    \end{figure}
    
    \begin{figure}[H]
        \centering
        \includegraphics[scale=0.2]{TCC/figuras/pl-lu.jpg}
        \caption{Gráficos polares de namorados.}
        \label{pl-namorados}
    \end{figure}
    
    \begin{figure}[H]
        \centering
        \includegraphics[scale=0.2]{TCC/figuras/hm-lu.jpg}
        \caption{Mapa de calor de namorados.}
        \label{hm-namorados}
    \end{figure}
  
  \item \textbf{Relacionamento profissional:}
    \begin{figure}[H]
        \centering
        \includegraphics[scale=0.45]{TCC/figuras/ds_renato.png}
        \caption{Estatísticas descritivas de colegas de trabalho.}
        \label{ds-trabalho}
    \end{figure}
    
    \begin{figure}[H]
        \centering
        \includegraphics[scale=0.4]{TCC/figuras/pz_renato.png}
        \caption{Gráficos de pizza de colegas de trabalho.}
        \label{pz-trabalho}
    \end{figure}
    
    \begin{figure}[H]
        \centering
        \includegraphics[scale=0.325]{TCC/figuras/bars_renato.png}
        \caption{Série temporal de colegas de trabalho.}
        \label{bars-trabalho}
    \end{figure}
    
    \begin{figure}[H]
        \centering
        \includegraphics[scale=0.325]{TCC/figuras/area_renato.png}
        \caption{Gráfico mensal de colegas de trabalho.}
        \label{area-trabalho}
    \end{figure}
    
    \begin{figure}[H]
        \centering
        \includegraphics[scale=0.4]{TCC/figuras/pl_renato.png}
        \caption{Gráficos polares de colegas de trabalho.}
        \label{pl-trabalho}
    \end{figure}
    
    \begin{figure}[H]
        \centering
        \includegraphics[scale=0.4]{TCC/figuras/hm_renato.png}
        \caption{Mapa de calor de colegas de trabalho.}
        \label{hm-trabalho}
    \end{figure}
  
  \item \textbf{Relacionamento familiar:}
    \begin{figure}[H]
        \centering
        \includegraphics[scale=0.45]{TCC/figuras/st_clara.png}
        \caption{Estatísticas descritivas de familiares.}
        \label{ds-familia}
    \end{figure}
    
    \begin{figure}[H]
        \centering
        \includegraphics[scale=0.4]{TCC/figuras/pz_clara.png}
        \caption{Gráficos de pizza de familiares.}
        \label{pz-familia}
    \end{figure}
    
    \begin{figure}[H]
        \centering
        \includegraphics[scale=0.325]{TCC/figuras/bars_clara.png}
        \caption{Série temporal de familiares.}
        \label{bars-familia}
    \end{figure}
    
    \begin{figure}[H]
        \centering
        \includegraphics[scale=0.325]{TCC/figuras/area_clara.png}
        \caption{Gráfico mensal de familiares.}
        \label{area-familia}
    \end{figure}
    
    \begin{figure}[H]
        \centering
        \includegraphics[scale=0.4]{TCC/figuras/pl_clara.png}
        \caption{Gráficos polares de familiares.}
        \label{pl-familia}
    \end{figure}
    
    \begin{figure}[H]
        \centering
        \includegraphics[scale=0.4]{TCC/figuras/hm_clara.png}
        \caption{Mapa de calor de familiares.}
        \label{hm-familia}
    \end{figure}
\end{enumerate}

Conforme é possível observar nas Figuras 4.1 até a Figura 4.18, existem diferenças consideráveis entre os gráficos e estatísticas. Essas diferenças serão examinadas nos tópicos abaixo:

\begin{itemize}
    \item Total de mensagens da conversa:\\
    A conversa de exemplo de namorados possui uma quantidade maior de mensagens enviadas quando comparada com as demais classes, apesar de ser a mais recente de todas. Isso pode ser explicado pela intensidade de um relacionamento amoroso, que costumeiramente envolve troca de mensagens diárias. Isso também é refletido na média de mensagens trocadas por dia, que é de 211 mensagens para o caso dos namorados, 11 para o relacionamento profissional e 5 no caso do relacionamento familiar.
    
    \item Quantidade de dias conversados:\\
    Percentual de dias em que houve conversa desde a primeira mensagem enviada até a data de exportação:\\
    \hspace{1cm}- Relacionamento amoroso: 84.8\%\\
    \hspace{1cm}- Relacionamento profissional: 29.5\%\\
    \hspace{1cm}- Relacionamento familiar: 28.51\%\\
    Conforme citado no item acima, pela dinâmica intrínseca de um relacionamento amoroso, a proporção de dias conversados é bem maior do que os outros casos analisados.
    
    \item Dia mais ativo:\\
    As estatísticas de cada classe de conversa utilizada como exemplo são:\\
    \hspace{1cm}- Relacionamento amoroso: 1.511 mensagens\\
    \hspace{1cm}- Relacionamento profissional: 124 mensagens\\
    \hspace{1cm}- Relacionamento familiar: 145 mensagens\\
    Essa quantidade destoante da quantidade de mensagens trocadas em um único dia no caso da conversa entre namorados pode ser explicada pelo fato de no início do relacionamento haver um período em que o casal se conhece melhor, passando muitas horas conversando seguidamente.
    
    \item \textit{Emojis} utilizados:\\
    No caso do relacionamento amoroso e do familiar, os \textit{emojis} enviados normalmente envolvem expressões de amor e de carinho. Para o caso do relacionamento entre colegas de trabalho, os \textit{emojis} utilizados são mais descontraídos, um fato interessante é que apenas um dos participantes desta conversa envia \textit{emojis}.

    \item Série temporal:\\
    Pode-se observar com facilidade os \textit{outliers}: para o caso do relacionamento amoroso, nota-se um pico evidente na metade do mês de agosto. Essa anomalia pode ser explicada por uma viagem que ocorreu no dia 7 de agosto em que os namorados ficaram separados por dez dias. Percebe-se que o mês de agosto foi marcado por essa distância extraindo essa informação do gráfico, pois fica nítida a quebra do padrão dos meses anteriores.
    
    \item Quantidade de mensagens por dia da semana:\\
    O \textit{chat} entre colegas de trabalho apresenta um pico de mensagens trocadas na segunda-feira, comportamento esperado já que não há expediente durante o final de semana. A conversa entre namorados tem uma queda de mensagens trocadas na sexta e no sábado, dias que normalmente eles estão juntos. Já para o caso da conversa entre mãe e filha (relacionamento familiar), há uma queda considerável das mensagens trocadas no final de semana, já que ambas estão juntas pois residem na mesma casa.
    
    \item Quantidade de mensagens por hora do dia:\\
    O gráfico destoante dos outros neste caso é o da conversa entre colegas de trabalho, é possível perceber claramente a concentração das mensagens durante o horário de expediente, entre 9h e 17h. Os outros exemplos possuem mensagens distribuídas ao longo de todo o dia.
\end{itemize}
\chapter{Conclusão}
\resumodocapitulo{Aqui analisam-se os resultados obtidos pelos algoritmos desenvolvidos neste trabalho, a fim de explorar o potencial da análise sugerida.}

O \href{https://analizap.tk}{Analizap} foi concebido com o propósito de extrair \textit{insights} e curiosidades sobre conversas entre duas pessoas no aplicativo WhatsApp.

% Referências bibliográficas
\bibliographystyle{config/abntex2/abntex2-num} % use este estilo para ABNT numérico
\renewcommand{\bibname}{REFERÊNCIAS BIBLIOGRÁFICAS}
\phantomsection
\addcontentsline{toc}{chapter}{REFERÊNCIAS BIBLIOGRÁFICAS}
\bibliography{referencias}

% Anexos
%\anexos

% Não retirar estes comandos
\makeatletter 
\renewcommand{\@makechapterhead}[1]{
  {\parindent \z@ \raggedleft \setfontarial\bfseries 
        \LARGE \thechapter. \space\space 
    \uppercase{#1}\par
    \vskip 40\p@
  }
}
\makeatother

\include{TCC/capitulos/anexo_explicacao_codigos}

\end{document}
