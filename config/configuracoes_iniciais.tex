% GRAU
\grau{Engenheiro}{Eletricista} \tipodemonografia{o}{Trabalho de Conclusão de Curso}

% TÍTULO
% Os comandos a seguir servem para definir o título do trabalho. Para evitar  que o latex defina automaticamente a quebra de linha, foram definidos um comando por linha. Desta forma o autor define como quer que o título seja dividido em várias linhas. O exemplo abaixo é para um título que ocupa três linhas. Observe que mesmo com a linha 4 não sendo utilizada, o comando \titulolinhaiv é chamado.

\titulolinhai{Análise exploratória de uma conversa }
\titulolinhaii{do WhatsApp entre duas pessoas}
\titulolinhaiii{}
\titulolinhaiv{}

% AUTORES
\autori{Aluno 1}
\autorii{Aluno 2}
\autoriii{}

% BANCA EXAMINADORA
% Os nomes dos membros da banca são definidos a seguir. Pode-se ter até 5 membros da banca, numerados de i a v (algarismos romanos).
% Para trabalhos com apenas um autor, deve-se usar \autorii{} para que não apareça um nome para segundo autor. É incumbência do usuário definir no argumento dos comandos a afiliação do membro da banca, assim como sua posição (se for orientador ou co-orientador). 
% Os nomes definidos pelos comandos abaixo aparecem na ordem de i a v.
\membrodabancai{Prof. 1, ENE/UnB}
\membrodabancaifuncao{Orientador}
\membrodabancaii{Prof. 2, ENE/UnB}
\membrodabancaiifuncao{Examinador interno}
\membrodabancaiii{Prof. 3, ENE/UnB}
\membrodabancaiiifuncao{Examinador interno}

% DATA DA DEFESA
\mes{Setembro}
\ano{2022}

% FICHA CATALOGRÁFICA
% Colocar o nome do autor como vai aparecer no catálogo. Último sobrenome primeiro, depois o nome e sobrenomes intermediários. Ex.: Borges, Geovany Araújo
\autorcatalogo{Oliveira, Matheus Noschang de; Braz, Rubens Saito Mira}
% Colocar o nome abreviado. Último sobrenome primeiro, depois as iniciais do nome e sobrenomes intermediários. Ex.: Borges, G.A.
\autorabreviadocatalogo{Oliveira, M.N.; Braz, R.S.M}

% PALAVRAS CHAVE
\palavraschavecatalogoi{Análise exploratória}
\palavraschavecatalogoii{WhatsApp}
\palavraschavecatalogoiii{Visualização de dados}
\palavraschavecatalogoiv{Python}

% NÚMERO DA PUBLICAÇÃO
% Fornecido pelo departamento após a defesa
%\publicacao{TCC-}

%Número de páginas da dissertação.
\numeropaginascatalogo{\pageref{LastPage}~p.}